\documentclass[12pt]{article}

% Math and symbol packages
\usepackage{amsmath}
\usepackage{amssymb}
\usepackage{derivative}

% Figure Packages
\usepackage{graphicx}
\usepackage{wrapfig}
\usepackage{epstopdf}
\usepackage{float}
\usepackage{subfigure}

% Formatting and Random Text Generation
\usepackage{inputenc}
\usepackage[left=2.54cm,right=2.54cm,top=2.54cm,bottom=2.54cm]{geometry}
\usepackage{lipsum}

% Header and indent packages
\usepackage{fancyhdr}
\usepackage{indentfirst}

% Create Title Section
\title{Electric Potential and Fields}
\author{Trevor Swan \\
Department of Physics, Case Western Reserve University \\
Cleveland, OH 44016-7079}
\date{2/11/2025}

% Create paragraph formatting
%\setlength{\parindent}{3em}

% Actual Lab content
\begin{document}
\pagestyle{fancy}
\fancyhf{}

% Load the title
\maketitle
\thispagestyle{fancy}
\renewcommand{\headrulewidth}{0pt}

% Set up Footers
\fancyfoot[L]{Trevor Swan}
\fancyfoot[C]{\thepage}
\fancyfoot[R]{Electric Potential and Fields}

% Abstract section of Report
\section{Abstract}
I have tested the behavior of electrostatic potential fields that result from various geometries through the use of a simple AC circuit, DMM, centimeter-specific water plate, and various conductive metal geometries. To measure the electric potential between positively and negatively charged electrodes $7.0\pm0.1$ cm from a center point, I measured voltage increments of ~1.3 V down the x-axis and plotted them against their position. I then used a fitting model derived from the electric potential between two points in space. This model did not agree with the data, yielding an electrode spacing of $1.41\pm0.66$ cm and $\chi^2_{DOF}=2698.57$. I then tested the electric field for the dipoles along the x-axis, using a slightly different model. Making use of the same data, the model provided a dipole spacing of $0.07\pm0.01$ m that agreed with the measured distance. This model, however, did not fit our data well either yielding $\chi^2_{DOF}=2233.33$. Next, I gathered data using two evenly spaced parallel metal plates centered $6.0\pm0.01$ cm from the origin on the y-axis. In this iteration of the experiment, I took data in a similar fashion, but then plotted the data with a linear fit. For the data in between the two plats, I measured $\chi^2_{DOF}=404.42$ using Origin. This value, while lower than the first experiment, still indicates that the data does not align well with theory. Using this same data, I also calculated the electric field along the y-axis to be $E=64.3\pm4.4$ V/m. Finally, I measured the electric field between these same parallel plates, but now with a conductive hollow cylinder centered on the origin. Based on principles of conductors, I predicted an average field of zero inside the cylinder. The data collected for this final iteration did not agree with this theory, yielding a value of $E=-20.00\pm0.03$ V/m. Based on the aforementioned data and techniques and models used in these experiments, I cannot conclude that the physical behavior of electric potential and fields aligns with the theoretical behavior in this scenarios.

\newpage

% Conclusion
\section{Conclusion}
The expected behavior of electric potential and fields has been theorized using models and equations outlined in the Analysis and Figures generated. In contrast to expected values derived from theory, the first and second models for electric dipoles yielded $\chi^2_{DOF}=2698.57$ and $\chi^2_{DOF}=2233.33$, respectively. Moreover, the two plates' model yielded $\chi^2_{DOF}=404.42$, and the Hollow Conductive Cylinder had an average field of $E=-20.00\pm0.03$, when it was theorized to be 0. Conductors should behave such that the electric field in the center of them is 0, by Gauss' law. As supported by our calculations, this was \textbf{not} the case. As for the other experiments, the large $\chi^2_{DOF}$ indicate that the models were not accurate fits for our data, thus driving the $\chi^2_{DOF}$ value much higher than its accepted value of 1. A prominent reason for these discrepancies is from the large fluctuations in our DMM's reader. During the experiment, we found that the readings would jump around frequently, eventually settling on some value. It was an oversight to suggest that the error of these measurements are so small, and hence adjusting the error bars of our preliminary measurements would most likely result in much smaller $\chi^2_{DOF}$ values. On top of that, the electric dipoles, plates, and hollow cylinder could interact with each other through their non-submerged halves. This interferes with the electric field greatly, and the only way to circumvent this is through using a full tank, which is not feasible for a small physics laboratory. Other than that, the widespread use of this equipment results in improper data being collected due to wear and tear from mistakes over the years. A final avenue for yielding more accurate results would be from the choice of better models for fitting. The only issue with this solution is that the models are derived from widely accepted theory, so changing them would suggest widespread failure of modern physics and its theories. Ultimately, we \textbf{cannot} conclude that the behavior of electric potential and fields agrees, within respective uncertainties, to our proposed models using the performed experiments.

\subsection{Acknowledgments}
I would like to thank Pratham Bhashyakarla, CWRU Department of Physics, for his help in obtaining the experimental data, preparing the figures, and checking my calculations.

\subsection{References}
\begin{enumerate}
    \item Driscoll, D., General Physics I: Mechanics Lab Manual, “Inclined Plane,” CWRU Bookstore, 2014.
\end{enumerate}

\end{document}
